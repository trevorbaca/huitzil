The locations of 
\raisebox{0.125 em}{\scalebox{0.45}{\circled{OB}}},
\raisebox{0.125 em}{\scalebox{0.45}{\circled{T0}}},
\raisebox{0.125 em}{\scalebox{0.45}{\circled{T+}}}
are exact.
The locations of
\raisebox{0.125 em}{\scalebox{0.45}{\circled{P1}}} --
\raisebox{0.125 em}{\scalebox{0.45}{\circled{P6}}}
and 
\raisebox{0.125 em}{\scalebox{0.45}{\circled{T1}}} --
\raisebox{0.125 em}{\scalebox{0.45}{\circled{T5}}}
can vary:
interpolate these points linearly (as distances) or interpolate them
exponentially (as pitches). Once you determine the locations of the 14 contact
points, keep them constant: the points outline a loving cartography of the
string (and the inflection points of flight).

The conceit of the second half of the piece has to do with the flight in the
inscription of the score: Huitzil, in Nahuatl, is Hummingbird. And it is
Hummingbird, the first Americans say, who can fly to the beyond, messenger-bird
of our dreams no matter their charge.

The cadenza separating the two halves of the piece --- the middle systems on
page two --- can be improvised. The cadenza separates dreams from flight. It
should be played in anticipation of joy.
